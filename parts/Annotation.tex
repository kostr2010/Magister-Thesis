\begin{abstract}

    \begin{center}
        \large{Разработка средств инструментации бинарного файла многоязыковой виртуальной машины.} \\
    \large\textit{Назаров Константин Олегович} \\[1 cm]

    Инструментация байткода является распространенной техникой для изменения поведения или аттрибутов программы. Эта техника может использоваться в целях профилировки, мониторинга, а также для динамического изменения поведения программы даже после сборки приложения. Существующие библиотеки для бинарной инструментации кода рассчитаны на инструментацию бинарных файлов для какого-либо языка программирования. В ходе данной работе был разработан фреймворк для работы с бинарными файлами многоязыковой платформы. Было проведено сравнение с существующими аналогами, а также показаны возможности инструментации кода, полученного из исходных файлов на разных языках.

    \vfill

    \end{center}

\end{abstract}
\newpage
