\section{Постановка задачи}
\label{sec:Chapter1} \index{Chapter1}

\sloppy

Данная работа ставит своей целью предоставить возможность манипуляции бинарным файлом многоязыковой платформы. В данной работе будет рассмотен вопрос о представлении языковых сущностей в рамках библиотеки, а также о разработке интерфейса для их инспекции и изменения. Помимо этого, данная работа затронет вопрос интероперабельности между различными языками. Будет предложена архитектура обобщенного пользовательского интерфейса для инспекции и модификации зависимостей между сущностями из разных языков. Вопрос манипуляции бинарными инструкциями, а также графом инструкций будет затронут лишь поверхностно.

Имея в виду области применения техники бинарной манипуляции, а также поставленную цель, можно сформулировать следующие задачи, решаемые в данной работе:

\begin{itemize}
    \item Спроектировать и разработать библиотеку для инспекции и манипуляции бинарными файлами платформы.
    \item В рамках данной библиотеки реализовать удобный интрефейс для написания сложных анализов тел функций.
    \item Предоставить единый интерфейс для использования интероперабельности между различными языками платформы.
    \item Предоставить общий интерфейс для работы с бинарными файлами полученными из разных языков.
    \item Поддержать возможность представления одного и того же языка разными кодировками бинарного файла, чтобы единообразно работать с бинарными файлами, получанными из одного исходного файла различными фронтендами.
\end{itemize}

Исходя из дальнейших планов развития платформы, к решению задачи были выдвинуты следующие требования:

\begin{itemize}
    \item Имплементированная библиотека должна быть общего назначения, чтобы позже на ее основе имплементировать специализированные фреймворки и инструменты.
    \item Имплементированная библиотека должна иметь возможность быть обернутой в высокоуровневые обертки для различных языков, чтобы предоставить пользователям платформы удобный интерфейс для инспекции и манипуляции их бинарными файлами.
    \item Интерфейс для работы с телами функций должен скрывать от пользователя наличие служебных инструкций платформы, чтобы облегчить использование библиотеки.
    \item Интефрейс для работы с бинарным файлом должен скрывать от пользователя детали имплементации бинарного файла, чтобы уменьшить зависимость пользователя от версионирования формата файла.
\end{itemize}

Как видно из постановки задач и требований к ним, в данной работе не будут рассмотрены вопросы расширения средств языков или их стандартной библиотеки.

\newpage
