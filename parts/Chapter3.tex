\section{Исследование и построение решения задачи}
\label{sec:Chapter3} \index{Chapter3}

\sloppy

В предыдущей главе были рассмотрены различные решения для задачи бинарной манипуляции. На основе их анализа составим более детальный план решения задач, обозначенных в \autoref{sec:Chapter1}.

Задача разработки библиотеки для инспекции и манипуляции бинарными файлами разбивается на две части: разработка интерфейса для работы с бинарными инструкциями и разработка интерфейса для работы с метаданными бинарного файла. как говорилось раньше, в данной работе подробно будет рассмотрена только вторая часть интерфейса, подробное описание которого будет в \autoref{sec:Chapter4}.

Задача реализации интерфейса, удобного для написания сложных анализов тел функций сводится к выбору промежуточного представления для тел функций и написанию интерфейса, имплементирующего его. Отдельной задачей является проектирование и имплементация механизма, обеспечивающего выполнение требования скрывать от пользователя наличие служебных инструкций платформы. Данные задачи можно решать непосредственно, и подробный процесс будет описан далее в \autoref{sec:Chapter4}.

Предоставление интерфейса для использования интероперабельности между языками платформы является самодостаточной задачей, решение которой будет представлено в \autoref{sec:Chapter4}.

Задача разработки общего интерфейса для бинарных файлов, полученных из исходных файлов написанных на разных языках, вместе с требованием поддержать возможность представления одного и того же языка разными кодировками бинарного файла, а также с требованием скрыть от пользователя детали имплементации бинарного файла говорит о том, что в разрабатываемой библиотеке должен использоваться уровень абстракции языковых сущностей. Таким образом, задача сводится к проектированию необходимых языковых абстракций и механизма овеществления различных кодировок в единое представление. Отдельной задачей является дизайн и имплементация интерфейса поверх выделенных абстракций. Для имплементации решения данной задачи был полезен анализ фреймворка \texttt{BCA}, интерфейс которого позволяет пользователю работать напрямую с языковыми понятиями, такими как наследование, избегая необходимости вникать в детали бинарного файла. Также был полезен анализ фреймворка \texttt{ASM}, который показал удобство ограниченного применения шаблона посетителя для анализа метаданных бинарного файла.

\newpage
