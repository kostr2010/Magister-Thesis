\section{Заключение}
\label{sec:Chapter5} \index{Chapter5}

В результате проделанной работы была разработана библиотека для манипуляции бинарными файлами многоязыковой платформы. Для достижения этой цели были сделаны следующие шаги:

\begin{itemize}
    \item Был проведем анализ сущесвтующих решений для других платформ. Для каждого из них были обозначены преимущества, недостатки, и полезные для данной работы аспекты.
    \item Был спроектирован и имплементирован интерфейс для работы с языковыми метаданными бинарного файла. Имплементированный интрефейс ползволяет пользователям писать как обобщенный код для работы со всеми бинарными файлами платформы, так и языко-специфичный код, работающий только с бинарными файлами, полученными из конкретных фронтендов.
    \item В рамках интерфейса для для работы с языковыми метаданными бинарного файла был спроектирован интерфейс для анализа и добавления внешних зависимостей с учетом возможности интероперабельности.
    \item Благодаря выбору абстракций уровня исходного языка в рамках интерфейса для для работы с языковыми метаданными удалось скрыть детали имплементации бинарного фала платформы и его фронтендов.
    \item Был спроектирован и имплементирован интерфейс для работы с телами функций. Имплементированный интерфейс предоставляет возможность писать сложные анализы, поскольку предоставляет код функций в \texttt{SSA} форме. Благодаря введенной системе \texttt{Target}-ов, данный интерфейс получилось сделать безопасным для использования, отделив все возможные наборы инструкций платформы от структуры графа, и привязав каждый набор инструкций к конкретным \texttt{Target}-ам.
\end{itemize}

На данный момент описанная библиотека еще находится на стадии разработки, и на данный момент имплементирован функционал, необходимый для написания пользовательских сценариев. В дальнейшем планируется продолжать имплементировать и развивать библиотеку. Первоочередной задачей является имплементация всех не реализованных на данный момент функций. После этого библиотека будет развиваться путем добавления новых языковых расширений оп мере развития платформы, а также путем поддержки уже существующих расширений. Еще одной целью является создание оберток для библиотеки на языках высокого уровня. Для увеличения производительности библиотеки планируется поддержать режим работы, основанный на шаблоне "посетитель", аналогично фреймворку \texttt{ASM}. Это уменьшит накладные расходы на создание оберток библиотеки, что позволит встроить данную библиотеку в платформу для динамической манипуляции бинарными файлами.

\newpage
